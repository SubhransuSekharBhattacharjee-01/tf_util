% You should title the file with a .tex extension (hw1.tex, for example)
\documentclass[11pt]{article}

\usepackage{hyperref}
\usepackage{amsmath}
\usepackage{mathtools}
\usepackage{amssymb}
\usepackage{wrapfig}
\usepackage{fancyhdr}
\usepackage{tikz-qtree}
\usepackage{tikz-qtree-compat}
\usepackage[normalem]{ulem}
\usepackage{tikz}
\usepackage{graphicx}
\DeclareMathOperator*{\argmin}{argmin}
\DeclareMathOperator*{\argmax}{argmax}

\oddsidemargin0cm
\topmargin-2cm     %I recommend adding these three lines to increase the 
\textwidth16.5cm   %amount of usable space on the page (and save trees)
\textheight23.5cm  

\newcommand{\question}[2] {\vspace{.25in} \hrule\vspace{0.5em}
\noindent{\bf #1: #2} \vspace{0.5em}
\hrule \vspace{.10in}}
\renewcommand{\part}[1] {\vspace{.10in} {\bf (#1)}}

\newcommand{\myname}{Sean Bittner}
\newcommand{\myandrew}{srb2201@columbia.edu}
\newcommand{\myhwnum}{12}

\setlength{\parindent}{0pt}
\setlength{\parskip}{5pt plus 1pt}
 
\DeclarePairedDelimiter\abs{\lvert}{\rvert}%
 %
\pagestyle{fancyplain}
\rhead{\fancyplain{}{\myname\\ \myandrew}}

\begin{document}

\medskip                        % Skip a "medium" amount of space
                                % (latex determines what medium is)
                                % Also try: \bigskip, \littleskip

\thispagestyle{plain}
\begin{center}                  % Center the following lines
{\Large Normalizing flows in tf\_utils} \\
Sean Bittner \\
February 8, 2019 \\
\end{center}

\textbf{Planar Flows}: \\
\[f(z) = z + h(w^\top z + b)\]
\[ \frac{d f(z)}{dz} = some expression \]

\textbf{Cholesky Product Flows}: \\
\[ f(z) = \text{vec}(L(z)L(z)^\top) \]

\[L(z) = \begin{bmatrix} e^{z_1} & 0 & ... &  0 \\ z_2 & e^{z_3} &  & \\ z_4 & z_5 & e^{z_6} &  & \\ ... & & & \\ \end{bmatrix}\]

\[\text{diag}(z) = \begin{bmatrix} z_1 \\ z_3 \\ z_6 \\ .. \end{bmatrix}, \text{diag}(L(z)) = \begin{bmatrix} e^{z_1} \\ e^{z_3} \\ e^{z_6} \\ .. \end{bmatrix} \]

\[vec(M) = \begin{bmatrix} M_{11} \\ M_{21} \\ M_{22} \\ M_{31} \\ ... \\ M_{DD} \end{bmatrix} \]

\[\begin{vmatrix} \frac{d L(z)}{dz} \end{vmatrix} = \prod_{d=1}^D \text{diag}(L(z))_d = \prod_{d=1}^D e^{\text{diag}(z)_d} \]

\[\log(\begin{vmatrix} \frac{d L(z)}{dz}) \end{vmatrix} = \log(\prod_{d=1}^D e^{\text{diag}(z)_d}) = \sum_{d=1}^D \text{diag}(z)_d \]

\[ \Sigma(z) = L(z) L(z)^\top \]
\[\begin{vmatrix} \frac{d \text{vec}(\Sigma(z))}{d L(z)} \end{vmatrix} = 2^D \prod_{d=1}^D L_{dd}^{D-d+1} \]

\[\log (\begin{vmatrix} \frac{d \text{vec}(\Sigma(z))}{d L(z)} \end{vmatrix}) = D \log(2) \sum_{d=1}^D (D-d+1) \log(\text{diag}(L(z))_d) \]
\[ =  D \log(2) \sum_{d=1}^D (D-d+1) \text{diag}(z)_d \]

\[\log(\begin{vmatrix} \frac{d f(z)}{dz} \end{vmatrix}) = \log(\begin{vmatrix} \frac{d \text{vec}(\Sigma(z))}{dL(z)} \frac{dL(z)}{dz} \end{vmatrix}) = \log(\begin{vmatrix} \frac{d \text{vec}(\Sigma(z))}{dL(z)} \end{vmatrix} \begin{vmatrix} \frac{dL(z)}{dz} \end{vmatrix}) =  \log(\begin{vmatrix} \frac{d \text{vec}(\Sigma(z))}{dL(z)} \end{vmatrix}) + \log( \begin{vmatrix} \frac{dL(z)}{dz} \end{vmatrix}) \]

\end{document}

